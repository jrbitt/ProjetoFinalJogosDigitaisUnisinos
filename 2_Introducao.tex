Na seção introdutória do trabalho, você deve escrever, no mínimo, as seguintes coisas a respeito do trabalho, cada uma delas ocupando de 1 a 3 parágrafos (preferivelmente) 

\subsection{Motivação}
\label{secao:motivacao}
Nestes primeiros parágrafos do artigo, você descreve o que motiva o desenvolvimento do trabalho (motivos sólidos baseados na pesquisa de referências, como deficiência no mercado a ser suprida, área pouco explorada, grande demanda do mercado por determinada tecnologia ou inovação)\cite{notes2002}.  \cite{AIGAMEDEV}

\subsection{Objetivo Geral}
\label{secao:objetivo_geral}
Aqui você deixa bem claro o objetivo principal do trabalho -- o que foi desenvolvido (ou no caso de PF1, o que está em desenvolvimento): ``Este trabalho apresenta o desenvolvimento do jogo \emph{Fellowsheep Adventures in Creepyland}, um jogo de simulação de ovelhas que possuem o poder de alterar as expressões faciais de seus pastores, que pensam ter controle sobre elas." 

Também deve ser feita uma rápida contextualização (conexão entre a motivação e o objetivo). ``Neste contexto, o jogo explora elementos de jogos Simulação como o Happy Farm\footnote{My Invention, 2019} e Sim City 2000\footnote{Maxis, 1999} combinados com o gênero de Aventura e utilizando no game design conceitos de expressões faciais propostos por Ekman~\cite{Ekman:1978}".

Exemplo de referência à Seção \ref{secao:objetivos_especificos}.

\subsection{Objetivos Específicos}
\label{secao:objetivos_especificos}
Lorem ipsum dolor sit amet, consectetuer adipiscing elit, sed diam nonummy nibh euismod tincidunt ut laoreet dolore magna aliquam erat volutpat. Ut wisi enim ad minim veniam, quis nostrud exercitation ullamcorper suscipit lobortis nisl ut aliquip ex ea commodo consequat. Duis autem vel eum iriure dolor in hendrerit in vulpu-tate velit esse molestie consequat, vel illum dolore eu feugiat nulla facilisis at vero eros et accumsan et iusto odio dignissim qui blan-dit praesent luptatum zzril delenit augue duis dolore te feugait nulla facilisi.

\subsection{Estrutura do Artigo}
\label{secao:estrutura_do_artigo}
O artigo está organizado da seguinte maneira: a Seção~\ref{secao:analise_pesquisa_de_mercado} apresenta um levantamento de jogos relacionados e questões de mercado relacionadas com o jogo em desenvolvimento. Logo após, a Seção~\ref{secao:jogo} apresenta (...), seguida da Seção~\ref{secao:desenvolvimento} que descreve em detalhes as decisões projetuais e tecnolõgicas mais importantes no processo de desenvolvimento do jogo. Em seguida, são apresentados resultados das avaliações realizadas na Seção~\ref{secao:testes}, a análise de resultados na Seção~\ref{secao:analise_de_resultados} e por último, são apresentadas considerações e trabalhos futuros na Seção~\ref{secao:consideracoes_finais}.